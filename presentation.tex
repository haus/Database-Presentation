\documentclass{beamer}

\mode<presentation> {\usetheme{Antibes} }

\usepackage{times}

\title{Betadases}

\author{
Haus \\
The CAT
}

\AtBeginSection[]
{
  \begin{frame}<beamer>
    \frametitle{Outline}
    \tableofcontents[currentsection]
  \end{frame}
}

\begin{document}

\begin{frame}
  \titlepage
\end{frame}

\section{Introduction}

\begin{frame}
  \frametitle{Database - What is it?}
  \begin{itemize}
    \item Way of storing and retrieiving data
    \item Way of relating data
    \item Not a spreadsheet
  \end{itemize}
\end{frame}

\begin{frame}
  \frametitle{Databases - What kinds?}
  \begin{itemize}
    \item Early Models
    \begin{itemize}
      \item Hierarchical
      \item Network
      \item Inverted File models
    \end{itemize}
    \item Object/Object-relational Model
    \item Relational/Entity-relational Model
  \end{itemize}
	\pause
	We'll be focusing on the Relational/Entity-relational Model.
\end{frame}

\begin{frame}
  \frametitle{What is a DBMS or RDBMS?}
Simply, DBMS is a database management system and RDBMS is a relational database management system. Some common examples of RDBMS include mysql, sqlite, postgresql, access, sql server, and oracle.
\end{frame}

\begin{frame}
  \frametitle{Why relational?}
It's called a relational database because the tables and data within a given database are defined by how they relate to each other. 

We'll get more into this with an example later.
\end{frame}

\begin{frame}
  \frametitle{SQL}
  \begin{itemize}
    \item Structured Query Language
    \item Not standard, although many SQL standards/versions exist
    \item SQL-86, SQL-89, SQL-92, SQL:1999, SQL:2003, SQL:2008
  \end{itemize}
	Other database query languages include datalog and object query language
\end{frame}

\begin{frame}
  \frametitle{More SQL}
  \begin{itemize}
    \item Data Definition Language (CREATE, DROP, ALTER)
    \item Data Manipulation Language (SELECT, INSERT, UPDATE, DELETE)
    \item Data Control Language (GRANT, REVOKE)
  \end{itemize}
\end{frame}

\section{An aside on set theory}

\begin{frame}
  \frametitle{How do databases work?}
  \begin{itemize}
	\item Queries operate on sets of tuples and return sets or bags of tuples
	\item Tables are sets of tuples
	\item So queries operate on tables (sets of tuples) and return...a set of tuples (a table)
  \end{itemize}
\end{frame}

\begin{frame}
  \frametitle{How do databases work?}
  \begin{itemize}
	\item A tuple is an ordered list of elements. An n-tuple is a tuple of n elements. (So a 3-tuple is a tuple of 3 elements). For example a 3-tuple could describe an event. $<$event name, start time, end time$>$ would be a 3-tuple that accomplishes that. $<>$ are used to delineate tuples.
	\item A set is a unique and unordered collection of elements. When compared for equality, sets with the same elements in any order are considered equal. So \{4, 5, 9\}, \{5, 4, 9\}, and \{5, 9, 4\} are all the same 3 element set.
	\item A bag, also a multiset, are sets that can have multiple instances of the same element. They are also unordered, so \{4, 4, 5\} and \{4, 5, 4\} are considered equal.
	\item Sets and bags are delineated with \{\}.
  \end{itemize}
\end{frame}

\section{Using Mysql}

\begin{frame}
  \frametitle{To the Console!}
	For this section we'll all work together to build a simple database for a blog.

So let's fire up mysql/postgresql and go to work.
\end{frame}

\begin{frame}
  \frametitle{Advanced Topics}
  \begin{itemize}
    \item Prepared Statements and SQL Injection
    \item Transactions
    \item Replication
    \item Migration
  \end{itemize}
\end{frame}

\end{document}
